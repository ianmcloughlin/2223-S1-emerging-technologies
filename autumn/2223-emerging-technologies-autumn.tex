\documentclass[a4paper]{tufte-handout}

\usepackage{ians_notes}

\title{Assessment: Emerging Technologies}
\author{ian.mcloughlin@atu.ie}
\date{Autumn 22/23}

\begin{document}
 
\maketitle

These are the instructions for the assessment of Emerging Technologies in Autumn 22/23.
These cover the full 100\% of the assessment for this module.


\section{Submission}

\begin{itemize}
  \item The deadline for submission is August 31\textsuperscript{st}, 2023. 
  \item Your whole submission must be in a single GitHub repository.
  \item Use the form on the Moodle page to submit your repository.
  \item Commits in GitHub on or before the deadline will be considered.\footnote{Once you have submitted your URL, you do not need to do anything other than commit to your repository and push the changes to GitHub.}
\end{itemize}


\section{What to submit}
This assessment has three overlapping components, as follows.

\newthought{GitHub Repository $(20\%)$:}
\begin{itemize}
  \item Create a single GitHub repository for all your work.
  \item Describe in a README the contents of the repository and how to run the files within it.
  \item Include an appropriate \texttt{.gitignore} file.
  \item Ensure regular and appropriate commits are made to the repository throughout your work on the project.
  \item Ensure filenames are clear and no necessary files are added.
\end{itemize}

\newthought{Qiskit Notebook $(40\%)$:}
\begin{itemize}
  \item Add a notebook to your repository named \texttt{Qiskit.ipynb}.
  \item In the notebook, explain what the \texttt{Qiskit} package is used for.
  \item Demonstrate the basic usage of \texttt{Qiskit}.
  \item Use visualizations to explain the core concepts of \texttt{Qiskit}.
\end{itemize}

\newthought{Deutsch Algorithm Notebook $(40\%)$:}
\begin{itemize}
  \item Add a notebook to your repository named \texttt{deutsch.ipynb}.
  \item In the notebook, explain the Deutsch algorithm\footnote{The Deutsch algorithm is the one qubit version of the Deutsch-Jozsa algorithm.}.
  \item Clearly define and explain the problem, and include an explanation of how to solve it on a classical computer.
  \item Simulate the solution of the problem on a quantum computer using \texttt{Qiskit}.
\end{itemize} 


\section{Marking Scheme}
Each component will be marked using the four categories below.
To receive a good mark in a category, your submission needs to provide evidence of meeting each of the criteria listed under it\footnote{In line with ATU policy, the examiners' overall impression of the submission may affect individual marks in each category.}.

\begin{description}
  \item[Research $(25\%)$:] evidence of research on topics; appropriate referencing; building on work of others; comparison to similar work.
  \item[Development $(25\%)$:] clear, concise, and correct code; appropriate tests; demonstrable knowledge of different approaches and algorithms; clean architecture.
  \item[Documentation $(25\%)$:] clear explanations of concepts in notebooks; concise comments in code and elsewhere; appropriate, standard README for a GitHub repository.
  \item[Consistency $(25\%)$:] tens of commits, each representing a reasonable amount of work; literature, documentation, and code evidencing work on the assessment; evidence of reviewing and refactoring.
\end{description}


\section{Advice}

\begin{itemize}
  \item Students sometimes struggle with the freedom given in an open-style assessment.
  \item You must decide where and how to start, what is relevant content for your submission, how much is enough, and how to make the submission your own.
  \item This is by design --- we assume you have a reasonable knowledge of programming and an ability to source your own information.
  \item Companies tell us they want graduates who can (within reason) take initiative, work independently, source information, and make design decisions without needing to ask for help.
  \item The point of this assessment is to demonstrate you can do that.
  \item You need a plan, you cannot just start coding straight away.
\end{itemize}


\section{Policies}

\begin{itemize}
  \item You are bound by all ATU policies and any GMIT policies that have not yet been replaced by new ATU policies.
  \item Review the GMIT Quality Assurance Framework~\cite{gmitqaf}.
  \item Pay particular attention to the Policy on Plagiarism and the Code of Student Conduct.
  \item If you have any doubts about what is permissible, email me to ask\footnote{\url{ian.mcloughlin@atu.ie}}.
\end{itemize}


\bibliographystyle{plainnat}
\nobibliography{bibliography}

\end{document}