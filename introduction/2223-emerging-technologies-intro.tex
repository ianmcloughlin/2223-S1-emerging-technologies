\documentclass[a4paper]{tufte-handout}

\usepackage{ians_notes}

\title{Emerging Technologies}
\author{ian.mcloughlin@atu.ie}
\date{Winter 22/23}

   
\begin{document}
 
\maketitle

An introduction to new and emerging technologies in computing.
Technologies such as new paradigms, new programming languages, new infrastructures, and new communications protocols will be investigated.


\section{Learning Outcomes}

On completion of this module the learner will/should be able to\footnote{These learning outcomes were recently updated and are making their way through our quality assurance procedures.}:

\begin{enumerate}
\item Detect new and emerging technologies in computing through reputable sources.
\item Contextualise an emerging technology by identifying its origins and proponents.
\item Research an emerging technology in order to use it.
\item Implement a solution to a computing problem using an emerging technology.
\end{enumerate}


\section{Assessment}

This assessment will be in the form of a portfolio\footnote{Here a portfolio will essentially mean a GitHub repository.} and is $100\%$ continuous assessment.

\begin{description}
  \item[$20\%$] Presentation of portfolio
  \item[$40\%$] Theory element of portfolio
  \item[$40\%$]	Practical element of portfolio
\end{description}

 
\section{Delivery}

This is a semester long module\footnote{Each semester typically has thirteen teaching weeks.}.
\begin{itemize}
  \item Realistically, we will have ten uninterrupted teaching weeks.
  \item There are many ideas about how lecturers should deliver modules.
  \item Some suggest a top-down, structured approach where topics are clearly defined ahead of time.
  \item Others suggest involving students in decisions, letting content evolve during the semester.
  \item Let's not be idealistic about it: we'll have an initial plan and tailor it during the semester.
  \item It is worth discussing in the Moodle forums what you as a class would like to work on.
  \item Just keep in mind that everyone will want something different.
  \item Also, remember that there is one lecturer and dozens of students in each of several modules.
  \item Time is limited, we will have to be careful about scope creep.
\end{itemize}
 
\section{Topics}

We will start with a plan to cover these five topics.

\begin{description}
  \item[GitHub Pages:] Hosting your own static site.
  \item[JupyterLite:] Web Assembly and browser storage.
  \item[JavaScript Frameworks:] Processing and Svelte. 
  \item[Fourier Transform:] Algorithms and applications.
  \item[Quantum Computing:] QisKit and Deutsch's algorithm.
\end{description}

\section{Advice}

Based on previous delivery of the module.

\begin{itemize}
  \item Everyone procrastinates, you need a strategy to compensate.
  \item You will be less stressed if work regularly, a bit every week.
  \item Review the marking scheme regularly and work to it.
  \item You must learn to deal with uncertainty in decision-making.
\end{itemize}

\section{Policies}
\begin{marginfigure}%
  \centering
  \includegraphics[width=0.6\linewidth]{img/atu-green.png}
  \caption*{GMIT is now ATU.}
  \label{fig:atulogo}
\end{marginfigure}
\begin{itemize}
  \item In April 2022, GMIT merged with IT Sligo and LyIT to become ATU, the Atlantic Technological University.
  \item Although the merger has happened, it will take a couple of years for our systems and policies to fully merge.
  \item During this time, we will continue to use GMIT's policies where an ATU policy has not yet superseded them.
  \item That means the GMIT Quality Assurance Framework~\cite{gmitqaf}.
\end{itemize} 


\bibliographystyle{plainnat}
\nobibliography{bibliography}

\end{document}