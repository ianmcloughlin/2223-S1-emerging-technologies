\documentclass[a4paper]{tufte-handout}

\usepackage{ians_notes}

\title{Introduction: Emerging Technologies}
\author{ian.mcloughlin@atu.ie}
\date{Winter 22/23}
   
\begin{document}

\maketitle

An introduction to new and emerging technologies in computing.
Technologies such as new paradigms, new programming languages, new infrastructures, and new communications protocols will be investigated.


\section{Learning Outcomes}

On completion of this module the learner will/should be able to\footnote{We recently updated these learning outcomes. They are making their way through our quality assurance procedures.}:

\begin{enumerate}
\item Detect new and emerging technologies in computing through reputable sources.
\item Contextualize an emerging technology by identifying its origins and proponents.
\item Research an emerging technology in order to use it.
\item Implement a solution to a computing problem using an emerging technology.
\end{enumerate}


\section{Assessment}

Assessment is in the form of a portfolio\footnote{The portfolio will be in the form of a GitHub repository demonstrating your work throughout the semester.}. There is no final exam.

\begin{description}
  \item[$20\%$] Presentation of portfolio
  \item[$40\%$] Theory element of portfolio
  \item[$40\%$]	Practical element of portfolio
\end{description}

 
\section{Delivery}

\begin{itemize}
  \item This is a semester long module. Realistically, we will have ten uninterrupted teaching weeks\footnote{Each semester typically has thirteen teaching weeks but some of those weeks have public holidays and other interruptions.}.
  \item There are many ideas about how lecturers should deliver modules. Some suggest a top-down, structured approach where topics are clearly defined ahead of time. Others suggest involving students in decisions, letting content evolve during the semester. Let's not be idealistic about it: we'll have an initial plan and tailor it during the semester.
  \item It is worth discussing in the Moodle forums what you as a class would like to work on. Just keep in mind that everyone will want something different. Also, remember that there is one lecturer and dozens of students in each of several modules. Time is limited, we will have to be careful about scope creep.
\end{itemize}

 
\section{Lectures and Practicals}

\begin{itemize}
  \item Traditionally, lectures covered theory and students applied the theory in practicals. That can feel a bit contrived and artificial, especially in computing where practice often comes before theory.
  \item We won't make a clear distinction between lectures and practicals where possible. Rather, we will focus on topics, projects, and problems.
  \item Notes will be in the form of Jupyter notebooks and practical work will be indicated in those notebooks.
\end{itemize}


\section{Topics}

We will start with a plan to cover these five topics.

\begin{description}
  \item[GitHub Pages:] Static sites and continuous integration.
  \item[JupyterLite:] Web Assembly and browser storage.
  \item[Computation:] New algroithms and old problems. 
  \item[Fourier Transform:] Algorithms and applications.
  \item[Quantum Computing:] QisKit and Deutsch's algorithm.
\end{description}


\section{Advice}

\begin{itemize}
  \item Everyone procrastinates, you need a strategy to compensate. You will be less stressed if you work regularly, a bit every week.
  \item Review the marking scheme regularly and work to it.
  \item Be able to demonstrate your work. This is easier for practical work, you often have code and the like. Theoretical work can be demonstrated through writing, images, plots, and diagrams.
  \item You will have to grapple with the uncertainty of making your own content and design decisions. That can be difficult.
\end{itemize}


\section{Policies}

\begin{marginfigure}%
  \centering
  \includegraphics[width=0.6\linewidth]{img/atu-green.png}
  \caption*{GMIT is now ATU.}
  \label{fig:atulogo}
\end{marginfigure}

\begin{itemize}
  \item In April 2022, GMIT merged with IT Sligo and LyIT to become ATU, the Atlantic Technological University.
  \item Although the merger has happened, it will take a couple of years for our systems and policies to fully merge.
  \item During this time, we will continue to use GMIT's policies where an ATU policy has not yet superseded them.
  \item That means the GMIT Quality Assurance Framework~\cite{gmitqaf}.
\end{itemize} 


\bibliographystyle{plainnat}
\nobibliography{bibliography}

\end{document}